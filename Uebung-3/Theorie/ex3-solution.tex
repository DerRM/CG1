\documentclass[12pt]{scrreprt}

\usepackage[ngerman]{babel}
\usepackage{textcomp}
\usepackage[T1]{fontenc}
\usepackage[utf8]{inputenc}

\usepackage{color,fancyvrb}
\usepackage{lmodern}

\usepackage{graphicx}
\usepackage{amssymb,amsmath}
%\graphicspath{{../latex/}}

%% Quellcode auszeichnung %%
\usepackage{minted}
\makeatletter
\minted@define@extra{label}
\makeatother
\definecolor{bg}{rgb}{0.95,0.95,0.95}
\usemintedstyle{manni}
\newminted{c}{fontfamily=helvetica, fontsize=\scriptsize, tabsize=8,
samepage, frame=single, bgcolor=bg,label=Quellcode,
numbersep=5pt, xrightmargin=0mm, xleftmargin=4mm, linenos}

\usepackage{url}
\usepackage{wrapfig}
\usepackage[font=small,labelfont=bf,textfont=sf]{caption}
\usepackage{placeins}
\usepackage{multirow}
\setlength{\parskip}{3ex plus 2ex minus 2ex}
\widowpenalty=300
\clubpenalty=300

\newcommand{\comment}[1]{}
\setcounter{secnumdepth}{6}

% header and footer
\usepackage{fancyhdr}
\setlength{\headheight}{18pt}
\renewcommand{\headrulewidth}{0.3pt}
\pagestyle{fancyplain}
\renewcommand{\chaptermark}[1]{\markboth{#1}{}}
\fancyhf{}
\cfoot{\thepage}

% make internal and external links
\usepackage{hyperref} % load this at the end of all packages, but
%before the options, not true here at the moment tough

\hypersetup{
%    bookmarks=true,         % show bookmarks bar? - is already true
    unicode=false,          % non-Latin characters in Acrobat’s bookmarks
    pdftoolbar=true,        % show Acrobat’s toolbar?
    pdfmenubar=true,        % show Acrobat’s menu?
    pdffitwindow=false,     % window fit to page when opened
    pdfstartview={FitH},    % fits the width of the page to the window
    pdftitle={CG 1 - Exercise},    % title
    pdfauthor={CG 1 - Gruppe 6},     % author
    pdfsubject={},   % subject of the document
    pdfcreator={CG 1 - Gruppe 6},   % creator of the document
    pdfproducer={Producer}, % producer of the document
    pdfkeywords={CG 1} {Computer Graphics I} {TU Berlin}, % list of keywords
    pdfnewwindow=true,      % links in new window
    colorlinks=false,       % false: boxed links; true: colored links
    linkcolor=red,          % color of internal links
    citecolor=green,        % color of links to bibliography
    filecolor=magenta,      % color of file links
    urlcolor=cyan           % color of external links
}

\begin{document}

\begin{titlepage}
  \begin{center}
    \vspace*{1cm}
    \textbf
    {\emph
      {\Large Computer Graphics I}}\\[1cm]
    \textbf
    {\Large Technische Universität Berlin}\\[.5cm]
    \textbf
    {\Large WS 2012/2013}\\[2cm]
    %\includegraphics[bb=0 0 60 40]{./TUB_Logo.png}\\[0.9cm]
    \vfill
    {\Large Gruppe 11}\\[1cm]
    Christopher Sierigk\\
    Bernd Loeber\\
    Silvio Hädrich\\
    \vspace*{2cm}
    \today
  \end{center}
  \vspace*{2cm}
\end{titlepage}

\pagenumbering{arabic}

\chapter*{Aufgabe 1 - Praktischer Teil}

siehe Quellcode

\chapter*{Aufgabe 2 - Theoriefragen}
\section*{1.}
Phänomenologie von Beleuchtungsmodellen.
\subsection*{(a)}
Ein Dreieck wird mittels Gouraud-Shading gezeichnet. Welches Bild ist zu erwarten, wenn eine Normale in Richtung Lichtquelle zeigt und die anderen Normalen davon abweichen. (0.5 Punkte)

In der Ecke, wo die Normale zur Lichtquelle zeigt trifft das Licht auf das Dreieck, was dadurch entsprechend aufgehellt wird (sehr hell). Von dort aus wird linear interpoliert, wodurch der Rest des Dreiecks von der Ecke aus gesehen immer dunkeler wird.

\subsection*{(b)}
Auf welche der Beleuchtungskomponenten ambient, difus, spekulär hat die Position des Betrachters bei einer ansonsten statischen Szene Einfluss? Begründen Sie Ihre Antwort. (0.5 Punkte)

Nur bei der spekulären Reflektion hat die Position des Betrachters Auswirkungen auf die Szene. Da der Lichtstrahl bei der Reflektionsform direkt und ohne Streuung von dem Objekt reflektiert wird, kann es sein, dass bei einem Positionswechsel des Betrachters die Reflektion des Lichtstrahls nicht wahrgenommen werden kann.

\section*{2.}
In 1978 führte Jim Blinn „bump mapping“ in der Computergrafik ein. Nennen Sie Vor- und Nachteile des Verfahrens. Warum wird es insbesondere im Bereich von Computerspielen häufig verwendet? (1 Punkt)
Vorteile:
-Verbessert die Qualität und den Realismus einer Darstellung, da deren Oberfläche strukturiert und nicht glatt erscheint
-Vergleichsweise schnell
-Reduziert die Anzahl der benötigten Polygone um eine ähnliche Darstellung hinzubekommen

Nachteile:
-Bei sehr flachen Betrachtungswinkeln kann man erkennen, dass keine wirkliche Struktur auf dem Objekt vorhanden ist
-Die Kanten von Objekten, die mit bump mapping bearbeitet wurden, sind weiterhin scharf und unstrukturiert

Bumpmapping wird wahrscheinlich in Computerspielen verwendet um Objekte mit relative geringem Mehraufwand besser aussehen zu lassen. Bei schnellen Bewegungen sollten die genannten Nachteile auch nicht sichtbar sein. Ein weiterer Grund könnten Restriktonen beim Speicherplatz sein, so dass man keine Aufwändigen Polygon Modelle speichern muss.

\section*{3.}
Erläutern Sie die Unterschiede zwischen uniform, attribute und varying Variablen in GLSL (1 Punkt)
Alle 3 sind In- und Outputs von Shadern und werden global deklariert.

Uniform Variablen verändern sich während dem Rendern nicht. Ein Beispiel sind Position und Farbe des Lichts. Sie sind readonly
Attribute Variablen werden bei Vertex Shadern genutzt und ändern die Vertices. Ein Beispiel sind Vertex Positionen und Normalen. Attributes sind ebenfalls readonly.
Varying Variablen werden zur Datenweitergabe von Vertex zum Fragment Shader genutzt. Sie können nur im Vertex Shader geschrieben werden, und sind in den Fragment Shadern readonly. Sie müssen im Vertex und Fragment Shader deklariert werden.


Uniforms werden benötigt, um Daten, welche konstant für alle Vertices
oder Fragmente sind, aus der Anwendung in die Shader hinein zu geben.
Dies sind z.B. die Matrizen oder entsprechende Berechnungsparameter wie
Materialeigenschaften oder Lichtdämpfung.

Als Attribute ausgezeichnete Variablen werden benutzt, um Werte pro
Vertex in den Vertex-Shader hinein zu geben. Dazu gehören Positions-
daten, Normale, Texturkoordinaten oder Farbwerte. Diese Daten werden
üblicherweise durch Buffer-Objekte bereit gestellt.

Bei Varyings handelt es sich um eine Art Container, um Werte aus
dem Vertex- in den Fragment-Shader zu transportieren. Wie im Kapitel
zur Rasterisierung bereits erwähnt wurde, werden die Werte für die Frag-
mente, welche sich zwischen den Vertex-Positionen befinden, interpoliert.
Wichtig ist, dass Name und Typ der deklarierten Variable in beiden Sha-
dern übereinstimmen. Eine Besonderheit für varyings ist, dass diese als in-
variant gekennzeichnet werden können. Dies kann bei multi-pass-Shadern
benutzt werden, um eventuelle Schwankungen aufgrund von Compileropti-
mierungen bei wiederholter Berechnung zu vermeiden.



\section*{4.}
Aus welchem Grund werden beim Aufbau eines BSP Baumes aus einem Dreicksnetz oftmals die enthaltenen Dreiecke selbst zum Spannen der Partitionierungsebene verwendet? Nennen sie mindestens eine weitere Möglichkeit, Partitionierungsebenen zu bestimmen. (1 Punkt)

Man verwendet die Dreiecke zum Spannen indem man die für die Splitting Planes die Planes der Dreiecke verwendet (Polygon-aligned). Dies ermöglicht ein genaues/eindeutiges Sortieren, ist aber auch vergleichsweise aufwändig.
Ausserdem ist dieses Vorgehen speichereffizient.

Andere Möglichkeit: Axis-Aligned: Hierbei wird die Szene mit einem Bounding Volume umschlossen, welches dann in immer kleinere Voulumes geteilt wird und so den BSP Baum bildet.
Man könnte auch anstatt der sich aus den vorhandenen Dreicken ergebenden Ebenen einfach beliebige Ebenen wählen.


\section*{5.}
In der Vorlesung wurde das Phong Shadingmodell diskutiert. Eine alternative Formulierung, ebenfalls von Blinn, verwendet den „half vector“.

\subsection*{(a)}
Was ist der half vector und welche Vorteile bietet er? (0.5 Punkte)
Der Halfvektor liegt in der Mitte des durch die Lichtquelle und Betrachterposition aufgespannten Winkels.

Bei diesem Verfahren muss, wenn Betrachter und Lichtquelle im Unendlichen liegen, muss der Halfvektor nur einmal berechnet werden, da dieser sich nicht ändert (Vorrausgesetzt Betrachter und Lichtquelle bleiben relativ gesehen an der gleichen Position).
\subsection*{(b)}
Leiten Sie das Phong Shadingmodell unter Verwendung des half vectors her. (0.5 Punkte)



\subsection*{(c)}
Was ist bei der Verwendung des half vectors zu beachten? (0.5 Punkte)

Dass das Skalarprodukt aus Halbierungsvektor und Normalenvektor > 0 sein muss.???
\subsection*{(d)}
Die „Half-Vektor“-Formulierung ist von der „Micro-Facet“-Model inspiriert. Erklären Sie kurz
das „Micro-Facet“-Model und warum es für physikalisch basiertes Rendering geeignet ist. (0.5
Punkte)

Bei dem Micro-Facet Model wird davon ausgegangen, dass die Oberfläche von objekten aus mikroskopisch kleinen Unebenheiten besteht, bei der jede Unebenheit das Licht gemäß ihrer eigenen Normalen reflektiert. Diese Technik ist für physikalisch basiertes Rendering geeignet, da alle Flächen in der Realität auch von mikroskopisch kleine Unebenheiten durchzogen sind.
\end{document}
