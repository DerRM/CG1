\documentclass[12pt]{scrreprt}

\usepackage[ngerman]{babel}
\usepackage{textcomp}
\usepackage[T1]{fontenc}
\usepackage[utf8]{inputenc}

\usepackage{color,fancyvrb}
\usepackage{lmodern}

\usepackage{graphicx}
\usepackage{amssymb,amsmath}
%\graphicspath{{../latex/}}

%% Quellcode auszeichnung %%
\usepackage{minted}
\makeatletter
\minted@define@extra{label}
\makeatother
\definecolor{bg}{rgb}{0.95,0.95,0.95}
\usemintedstyle{manni}
\newminted{c}{fontfamily=helvetica, fontsize=\scriptsize, tabsize=8,
samepage, frame=single, bgcolor=bg,label=Quellcode,
numbersep=5pt, xrightmargin=0mm, xleftmargin=4mm, linenos}

\usepackage{url}
\usepackage{wrapfig}
\usepackage[font=small,labelfont=bf,textfont=sf]{caption}
\usepackage{placeins}
\usepackage{multirow}
\setlength{\parskip}{3ex plus 2ex minus 2ex}
\widowpenalty=300
\clubpenalty=300

\newcommand{\comment}[1]{}
\setcounter{secnumdepth}{6}

% header and footer
\usepackage{fancyhdr}
\setlength{\headheight}{18pt}
\renewcommand{\headrulewidth}{0.3pt}
\pagestyle{fancyplain}
\renewcommand{\chaptermark}[1]{\markboth{#1}{}}
\fancyhf{}
\cfoot{\thepage}

% make internal and external links
\usepackage{hyperref} % load this at the end of all packages, but
%before the options, not true here at the moment tough

\hypersetup{
%    bookmarks=true,         % show bookmarks bar? - is already true
    unicode=false,          % non-Latin characters in Acrobat’s bookmarks
    pdftoolbar=true,        % show Acrobat’s toolbar?
    pdfmenubar=true,        % show Acrobat’s menu?
    pdffitwindow=false,     % window fit to page when opened
    pdfstartview={FitH},    % fits the width of the page to the window
    pdftitle={CG 1 - Exercise},    % title
    pdfauthor={CG 1 - Gruppe 6},     % author
    pdfsubject={},   % subject of the document
    pdfcreator={CG 1 - Gruppe 6},   % creator of the document
    pdfproducer={Producer}, % producer of the document
    pdfkeywords={CG 1} {Computer Graphics I} {TU Berlin}, % list of keywords
    pdfnewwindow=true,      % links in new window
    colorlinks=false,       % false: boxed links; true: colored links
    linkcolor=red,          % color of internal links
    citecolor=green,        % color of links to bibliography
    filecolor=magenta,      % color of file links
    urlcolor=cyan           % color of external links
}

\begin{document}

\begin{titlepage}
  \begin{center}
    \vspace*{1cm}
    \textbf
    {\emph
      {\Large Computer Graphics I}}\\[1cm]
    \textbf
    {\Large Technische Universität Berlin}\\[.5cm]
    \textbf
    {\Large WS 2012/2013}\\[2cm]
    %\includegraphics[bb=0 0 60 40]{./TUB_Logo.png}\\[0.9cm]
    \vfill
    {\Large Gruppe 11}\\[1cm]
    Christopher Sierigk\\
    Bernd Loeber\\
    Silvio Hädrich\\
    \vspace*{2cm}
    \today
  \end{center}
  \vspace*{2cm}
\end{titlepage}

\pagenumbering{arabic}

\chapter*{Aufgabe 1 - Praktischer Teil}

siehe Quellcode

\chapter*{Aufgabe 2 - Theoriefragen}
\section*{1.}
Phänomenologie von Beleuchtungsmodellen.
\subsection*{(a)}
Ein Dreieck wird mittels Gouraud-Shading gezeichnet. Welches Bild ist zu erwarten, wenn eine Normale in Richtung Lichtquelle zeigt und die anderen Normalen davon abweichen. (0.5 Punkte)

In der Ecke, wo die Normale zur Lichtquelle zeigt trifft das Licht direkt auf, dadurch findet eine entsprechend starke Aufhellung statt (sehr hell). Die Helligkeit an den anderen Eckpunkt ist geringer. Dazwischen wird linear interpoliert, wodurch der Rest des Dreiecks von der Ecke (mit der Normalen in Richtung Lichtquelle) aus gesehen immer dunkler wird.

\subsection*{(b)}
Auf welche der Beleuchtungskomponenten ambient, diffus, spekulär hat die Position des Betrachters bei einer ansonsten statischen Szene Einfluss? Begründen Sie Ihre Antwort. (0.5 Punkte)

Die Position des Betrachters hat zum einen Einfluss auf den spekulären Anteil der Beleuchtung, weiterhin wird die diffuse Reflektion unterteilt in ideal diffus und gerichtet diffus. Auch die gerichtet diffuse Reflektion hängt von der Position des Betrachters ab, denn auch hier wird bei der Berechnung die Position des Betrachters einkalkuliert. Bei diesen beiden Beleuchtungskomponenten kommt es darauf an, wieviel Licht aus der Quelle von der Oberfläche eines Objektes in Richtung des Betrachters reflektiert wird.
Der ideal diffuse Anteil hingegen ist, ebenso wie der ambiente, unabhängig vom Betrachter.
>>>>>>> Stashed changes


\section*{2.}
In 1978 führte Jim Blinn „bump mapping“ in der Computergrafik ein. Nennen Sie Vor- und Nachteile des Verfahrens. Warum wird es insbesondere im Bereich von Computerspielen häufig verwendet? (1 Punkt)

\subsubsection*{Vorteile:}

\begin{itemize}
  \item deutliche Verbesserung des visuellen Eindruckes, da die
    Plastizität der dargestellten Objekte verbessert wird
  \item sehr schnell, da keine zusätzliche Geometrie erzeugt werden muss
  \item benötigt nur geringe Ressourcen
\end{itemize}

\subsubsection*{Nachteile:}

\begin{itemize}
  \item An den Silhouetten ist erkennbar, das die Objektstruktur noch immer flach ist
  \item keine Selbstverschattung
  \item keine Selbstokklusion
\end{itemize}


Bumpmapping wird in Computerspielen verwendet um Objekte mit relative geringem Mehraufwand besser aussehen zu lassen. Bei schnellen Bewegungen sollten die genannten Nachteile auch nicht sichtbar sein.
Ein weiterer Grund könnten Restriktionen beim Speicherplatz und Rechenleistung sein, so dass man keine Aufwändigen Polygon Modelle speichern muss. Somit kann auch noch auf vergleichsweise schwacher Hardware ein
zufriedenstellendes Ergebnis erzielt werden. Damit steigt die Zahl
potentieller Kunden für ein Spiel und somit der Gewinn des Unternehmens.


\section*{3.}
Erläutern Sie die Unterschiede zwischen uniform, attribute und varying Variablen in GLSL (1 Punkt)

Uniforms werden benötigt, um Daten, welche konstant für alle Vertices oder Fragmente sind, aus der Anwendung in die Shader hinein zu geben. Dies sind z.B. die Matrizen oder entsprechende Berechnungsparameter wie Materialeigenschaften oder Lichtdämpfung. Uniform Variablen verändern sich während dem Rendern nicht. Sie sind readonly.


Als Attribute ausgezeichnete Variablen werden benutzt, um Werte pro Vertex in den Vertex-Shader hinein zu geben. Dazu gehören Positionsdaten, Normale, Texturkoordinaten oder Farbwerte. Diese Daten werden üblicherweise durch Buffer-Objekte bereit gestellt. Attributes sind ebenfalls readonly.

Bei Varyings handelt es sich um eine Art Container, um Werte aus dem Vertex- in den Fragment-Shader zu transportieren.
Sie werden im Vertex Shader geschrieben und sind in den Fragment Shadern readonly.
Die Werte werden für die Fragmente, welche sich zwischen den Vertex-Positionen befinden, interpoliert.
Wichtig ist, dass Name und Typ der deklarierten Variable in beiden Shadern übereinstimmen.
Eine Besonderheit für varyings ist, dass diese als invariant gekennzeichnet werden können.
Dies kann bei multi-pass-Shadern benutzt werden, um eventuelle Schwankungen aufgrund von Compileroptimierungen bei wiederholter Berechnung zu vermeiden.


\section*{4.}
Aus welchem Grund werden beim Aufbau eines BSP Baumes aus einem Dreicksnetz oftmals die enthaltenen Dreiecke selbst zum Spannen der Partitionierungsebene verwendet? Nennen sie mindestens eine weitere Möglichkeit, Partitionierungsebenen zu bestimmen. (1 Punkt)

Man verwendet die Dreiecke zum Spannen indem man die für die Splitting Planes die Planes der Dreiecke verwendet (Polygon-aligned). Dies ermöglicht ein genaues/eindeutiges Sortieren, ist aber auch vergleichsweise aufwändig.
Ausserdem ist dieses Vorgehen speichereffizient.

Andere Möglichkeit: Axis-Aligned: Hierbei wird die Szene mit einem Bounding Volume umschlossen, welches dann in immer kleinere Volumes geteilt wird und so den BSP Baum bildet.

\section*{5.}
In der Vorlesung wurde das Phong Shadingmodell diskutiert. Eine alternative Formulierung, ebenfalls von Blinn, verwendet den „half vector“.

\subsection*{(a)}
Was ist der half vector und welche Vorteile bietet er? (0.5 Punkte)

Der Halfvektor liegt in der Mitte des durch die Lichtquelle und Betrachterposition aufgespannten Winkels.

\begin{itemize}
  \item die Berechnung vereinfacht sich, wenn Beobachter und
    Lichtquelle in der Unendlichkeit angenommen werden, es reicht dann
    eine einmalige Berechnung
  \item das Beleuchtungsresultat entspricht besser dem Original im
    Bezug auf das Reflexionsverhalten von Oberflächen eines Materials
\end{itemize}

\subsection*{(b)}
Leiten Sie das Phong Shadingmodell unter Verwendung des half vectors her. (0.5 Punkte)

Beim Phong Beleuchtungsmodel berechnet sich die spekuläre Leuchtdichte \(D_s\)
nach folgender Formel:
\[
D_s = r_s * D * cos^m \gamma
\]
\(\gamma\) kann unter der Vorraussetzung, dass sich Beobachtungsrichtung,
Lichtquellenrichtung, Normalenvektor der reflektierenden Fläche und
Reflexionsrichtung in einer Ebene befinden, berechnet werden mit:
\[
\gamma = H * N, wobei:  H = \frac{L + V}{|L + V|}
\]

\begin{itemize}
  \item \(r_s\) der Reflexionsgrad,
  \item D die Leuchtdichte der Lichtquelle,
  \item N die normalisierte Normale der Ebene,
  \item L der normalisierte Vektor in Lichtquellenrichtung und
  \item V der normalisierte Vektor in Richtung des Beobachters ist
\end{itemize}


\subsection*{(c)}
Was ist bei der Verwendung des half vectors zu beachten? (0.5 Punkte)

\begin{itemize}
  \item siehe oben, die Vektoren muessen in einer Ebene liegen
  \item man muss mit Einheitsvektoren rechnen
  \item das Ergebnis bei Verwendung des half vectors unterscheidet
    sich etwas vom Ergebnis bei Phong Shading
\end{itemize}


\subsection*{(d)}
Die „Half-Vektor“-Formulierung ist von der „Micro-Facet“-Model inspiriert. Erklären Sie kurz
das „Micro-Facet“-Model und warum es für physikalisch basiertes Rendering geeignet ist. (0.5
Punkte)

Bei dem Micro-Facet Model wird davon ausgegangen, dass die Oberfläche von objekten aus mikroskopisch kleinen Unebenheiten besteht, bei der jede Unebenheit das Licht gemäß ihrer eigenen Normalen reflektiert. Diese Technik ist für physikalisch basiertes Rendering geeignet, da alle Flächen in der Realität auch von mikroskopisch kleine Unebenheiten durchzogen sind.
\end{document}
