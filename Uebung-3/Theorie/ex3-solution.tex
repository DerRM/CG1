\documentclass[12pt]{scrreprt}

\usepackage[ngerman]{babel}
\usepackage{textcomp}
\usepackage[T1]{fontenc}
\usepackage[utf8]{inputenc}

\usepackage{color,fancyvrb}
\usepackage{lmodern}

\usepackage{graphicx}
\usepackage{amssymb,amsmath}
%\graphicspath{{../latex/}}

%% Quellcode auszeichnung %%
\usepackage{minted}
\makeatletter
\minted@define@extra{label}
\makeatother
\definecolor{bg}{rgb}{0.95,0.95,0.95}
\usemintedstyle{manni}
\newminted{c}{fontfamily=helvetica, fontsize=\scriptsize, tabsize=8,
samepage, frame=single, bgcolor=bg,label=Quellcode,
numbersep=5pt, xrightmargin=0mm, xleftmargin=4mm, linenos}

\usepackage{url}
\usepackage{wrapfig}
\usepackage[font=small,labelfont=bf,textfont=sf]{caption}
\usepackage{placeins}
\usepackage{multirow}
\setlength{\parskip}{3ex plus 2ex minus 2ex}
\widowpenalty=300
\clubpenalty=300

\newcommand{\comment}[1]{}
\setcounter{secnumdepth}{6}

% header and footer
\usepackage{fancyhdr}
\setlength{\headheight}{18pt}
\renewcommand{\headrulewidth}{0.3pt}
\pagestyle{fancyplain}
\renewcommand{\chaptermark}[1]{\markboth{#1}{}}
\fancyhf{}
\cfoot{\thepage}

% make internal and external links
\usepackage{hyperref} % load this at the end of all packages, but
%before the options, not true here at the moment tough

\hypersetup{
%    bookmarks=true,         % show bookmarks bar? - is already true
    unicode=false,          % non-Latin characters in Acrobat’s bookmarks
    pdftoolbar=true,        % show Acrobat’s toolbar?
    pdfmenubar=true,        % show Acrobat’s menu?
    pdffitwindow=false,     % window fit to page when opened
    pdfstartview={FitH},    % fits the width of the page to the window
    pdftitle={CG 1 - Exercise},    % title
    pdfauthor={CG 1 - Gruppe 6},     % author
    pdfsubject={},   % subject of the document
    pdfcreator={CG 1 - Gruppe 6},   % creator of the document
    pdfproducer={Producer}, % producer of the document
    pdfkeywords={CG 1} {Computer Graphics I} {TU Berlin}, % list of keywords
    pdfnewwindow=true,      % links in new window
    colorlinks=false,       % false: boxed links; true: colored links
    linkcolor=red,          % color of internal links
    citecolor=green,        % color of links to bibliography
    filecolor=magenta,      % color of file links
    urlcolor=cyan           % color of external links
}

\begin{document}

\begin{titlepage}
  \begin{center}
    \vspace*{1cm}
    \textbf
    {\emph
      {\Large Computer Graphics I}}\\[1cm]
    \textbf
    {\Large Technische Universität Berlin}\\[.5cm]
    \textbf
    {\Large WS 2012/2013}\\[2cm]
    %\includegraphics[bb=0 0 60 40]{./TUB_Logo.png}\\[0.9cm]
    \vfill
    {\Large Gruppe 11}\\[1cm]
    Christopher Sierigk\\
    Bernd Loeber\\
    Silvio Hädrich\\
    \vspace*{2cm}
    \today
  \end{center}
  \vspace*{2cm}
\end{titlepage}

\pagenumbering{arabic}

\chapter*{Aufgabe 1 - Praktischer Teil}

siehe Quellcode

\chapter*{Aufgabe 2 - Theoriefragen}
\section*{1.}
Phänomenologie von Beleuchtungsmodellen.
\subsection*{(a)}
Ein Dreieck wird mittels Gouraud-Shading gezeichnet. Welches Bild ist zu erwarten, wenn eine Normale in Richtung Lichtquelle zeigt und die anderen Normalen davon abweichen. (0.5 Punkte)
An der Ecke, wo die Normale zur Lichtquelle zeigt trifft das Licht auf das Dreieck, was dadurch entsprechend aufgehellt wird (sehr hell). Von dort aus wird interpoliert, wodurch der Rest des Dreiecks von der Ecke aus gesehen immer dunkeler wird.

\subsection*{(b)}
Auf welche der Beleuchtungskomponenten ambient, difus, spekulär hat die Position des Betrachters bei einer ansonsten statischen Szene Einfluss? Begründen Sie Ihre Antwort. (0.5 Punkte)

Nur bei der spekulären Reflektion hat die Position des Betrachters Auswirkungen auf die Szene. Da der Lichtstrahl bei der Reflektionsform direkt und ohne Streuung von dem Objekt reflektiert wird, kann es sein, dass bei einem Positionswechsel des Betrachters die Reflektion des Lichtstrahls nicht wahrgenommen werden kann.

\section*{2.}
In 1978 führte Jim Blinn „bump mapping“ in der Computergrafik ein. Nennen Sie Vor- und Nachteile des Verfahrens. Warum wird es insbesondere im Bereich von Computerspielen häufig verwendet? (1 Punkt)
Vorteile:
-Verbessert die Qualität und den Realismus einer Darstellung, da deren Oberfläche strukturiert und nicht glatt erscheint
-Vergleichsweise schnell
-Reduziert die Anzahl der benötigten Polygone um eine ähnliche Darstellung hinzubekommen

Nachteile:
-Bei sehr flachen Betrachtungswinkeln kann man erkennen, dass keine wirkliche Struktur auf dem Objekt vorhanden ist
-Die Kanten von Objekten, die mit bump mapping bearbeitet wurden, sind weiterhin scharf und unstrukturiert

Bumpmapping wird wahrscheinlich in Computerspielen verwendet um Objekte mit relative geringem Mehraufwand besser aussehen zu lassen. Bei schnellen Bewegungen sollten die genannten Nachteile auch nicht sichtbar sein. Ein weiterer Grund könnten Restriktonen beim Speicherplatz sein, so dass man keine Aufwändigen Polygon Modelle speichern muss.

\section*{3.}
Erläutern Sie die Unterschiede zwischen uniform, attribute und varying Variablen in GLSL (1 Punkt)
Alle 3 sind In- und Outputs von Shadern und werden global deklariert.

Uniform Variablen verändern sich während dem Rendern nicht. Ein Beispiel sind Position und Farbe des Lichts. Sie sind readonly
Attribute Variablen werden bei Vertex Shadern genutzt und ändern die Vertices. Ein Beispiel sind Vertex Positionen und Normalen. Attributes sind ebenfalls readonly.
Varying Variablen werden zur Kommunika
Varyings are used for passing data from a vertex shader to a fragment shader. Varyings are (perspective correct) interpolated across the primitive. Varyings are read-only in fragment shader but are read- and writeable in vertex shader (but be careful, reading a varying type before writing to it will return an undefined value). If you want to use varyings you have to declare the same varying in your vertex shader and in your fragment shader.

\section*{4.}
Aus welchem Grund werden beim Aufbau eines BSP Baumes aus einem Dreicksnetz oftmals die enthaltenen Dreiecke selbst zum Spannen der Partitionierungsebene verwendet? Nennen sie mindestens eine weitere Möglichkeit, Partitionierungsebenen zu bestimmen. (1 Punkt)
Man verwendet die Dreiecke zum Spannen indem man die für die Splitting Planes die Planes der Dreiecke verwendet (Polygon-aligned). Dies ermöglicht ein genaues/eindeutiges Sortieren, ist aber auch vergleichsweise aufwändig.

Andere Möglichkeit Axis-Aligned: Hierbei wird die Szene mit einem Bounding Voulume umschlossen, welches dann in immer kleinere Voulumes geteilt wird und so den BSP Baum bildet.

\section*{5.}
In der Vorlesung wurde das Phong Shadingmodell diskutiert. Eine alternative Formulierung, ebenfalls von Blinn, verwendet den „half vector“.

\subsection*{(a)}
Was ist der half vector und welche Vorteile bietet er? (0.5 Punkte)
\subsection*{(b)}
Leiten Sie das Phong Shadingmodell unter Verwendung des half vectors her. (0.5 Punkte)
\subsection*{(c)}
Was ist bei der Verwendung des half vectors zu beachten? (0.5 Punkte)
\subsection*{(d)}
Die „Half-Vektor“-Formulierung ist von der „Micro-Facet“-Model inspiriert. Erklären Sie kurz
das „Micro-Facet“-Model und warum es für physikalisch basiertes Rendering geeignet ist. (0.5
Punkte)
\end{document}
