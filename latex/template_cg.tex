\documentclass[12pt]{scrreprt}

\usepackage[ngerman]{babel}
\usepackage{textcomp}
\usepackage[T1]{fontenc}
\usepackage[utf8]{inputenc}

\usepackage{color,fancyvrb}
\usepackage{lmodern}

\usepackage{graphicx}
\usepackage{amssymb,amsmath}
%\graphicspath{{./images/}}

%% Quellcode auszeichnung %%
\usepackage{minted}
\makeatletter
\minted@define@extra{label}
\makeatother
\definecolor{bg}{rgb}{0.95,0.95,0.95}
\usemintedstyle{manni}
\newminted{c}{fontfamily=helvetica, fontsize=\scriptsize, tabsize=8,
samepage, frame=single, bgcolor=bg,label=Quellcode,
numbersep=5pt, xrightmargin=0mm, xleftmargin=4mm, linenos}

\usepackage{url}
\usepackage{wrapfig}
\usepackage[font=small,labelfont=bf,textfont=sf]{caption}
\usepackage{placeins}
\usepackage{multirow}
\setlength{\parskip}{3ex plus 2ex minus 2ex}
\widowpenalty=300
\clubpenalty=300

\newcommand{\comment}[1]{}
\setcounter{secnumdepth}{6}

% header and footer
\usepackage{fancyhdr}
\setlength{\headheight}{18pt}
\renewcommand{\headrulewidth}{0.3pt}
\pagestyle{fancyplain}
\renewcommand{\chaptermark}[1]{\markboth{#1}{}}
\fancyhf{}
\cfoot{\thepage}

% make internal and external links
\usepackage{hyperref} % load this at the end of all packages, but
%before the options, not true here at the moment tough

\hypersetup{
%    bookmarks=true,         % show bookmarks bar? - is already true
    unicode=false,          % non-Latin characters in Acrobat’s bookmarks
    pdftoolbar=true,        % show Acrobat’s toolbar?
    pdfmenubar=true,        % show Acrobat’s menu?
    pdffitwindow=false,     % window fit to page when opened
    pdfstartview={FitH},    % fits the width of the page to the window
    pdftitle={CG 1 - Exercise},    % title
    pdfauthor={CG 1 - Gruppe 6},     % author
    pdfsubject={},   % subject of the document
    pdfcreator={CG 1 - Gruppe 6},   % creator of the document
    pdfproducer={Producer}, % producer of the document
    pdfkeywords={CG 1} {Computer Graphics I} {TU Berlin}, % list of keywords
    pdfnewwindow=true,      % links in new window
    colorlinks=false,       % false: boxed links; true: colored links
    linkcolor=red,          % color of internal links
    citecolor=green,        % color of links to bibliography
    filecolor=magenta,      % color of file links
    urlcolor=cyan           % color of external links
}

\begin{document}

\begin{titlepage}
  \begin{center}
    \vspace*{1cm}
    \textbf
    {\emph
      {\Large Computer Graphics I}}\\[1cm]
    \textbf
    {\Large Technische Universität Berlin}\\[.5cm]
    \textbf
    {\Large WS 2012/2013}\\[2cm]
    %\includegraphics[bb=0 0 60 40]{./TUB_Logo.png}\\[0.9cm]
    \vfill
    {\Large Gruppe 11}\\[1cm]
    Christopher Sierigk\\
    Bernd Loeber\\
    Silvio Hädrich\\
    \vspace*{2cm}
    \today
  \end{center}
  \vspace*{2cm}
\end{titlepage}

\pagenumbering{arabic}

\chapter*{Aufgabe 1 - Praktischer Teil}

siehe Quellcode

\chapter*{Aufgabe 2 - Theoriefragen}
\section*{1.}

\subsection*{a.}

\subsection*{b.}

\subsection*{c.}

Weil in der Computergrafik sehr viel mit linearen Transformationen
gearbeitet wird, z.B. bei der Abbildung von 3-dimensionalen
Geometriemodellen in den 2D-Raum eines Bildschirmes.

Würde die geforderte Vorraussetzung nicht gelten, würden sich 2 sich
schneidende Linien unter Umständen nach der linearen Transformation
nicht mehr schneiden. Das wäre schlecht.


\section*{2. Geometrie Shader}

\subsection*{Funktionsweise}

Geometrieshader werden in der Grafikpipeline nach den Vertexshadern
ausgeführt.
Im Gegensatz zu Vertex- und Fragmentshadern, welche nur für die
Darstellung gegebener Geometrie geeignet sind, kann ein
Geometrieshader neue Vertices erstellen und damit die Geometrie von
Objekten verändern.

\subsection*{warum nicht schon frueher?}

Vermutlich war die Hardware noch nicht in der Lage dazu.
Vielleicht waren auch die Grafikapis (OpenGL, Directx) noch nicht
soweit fortgeschritten, das sie diese Möglichkeit konzeptionell
berücksichtigten.

\subsection*{Vorteile dieses Shaders?}

Im Gegensatz zu Vertex- und Fragmentshader welche nur ein Vertex bzw.
Pixel auf einmal bearbeiten können, hat man in einem Geometrieshader
auch zugriff auf Nachbarvertices.
Die Geometrie von Objekten kann on-the-fly verändert werden, d.h. es
können zusätzliche Details hinzugefügt werden. Die CPU wird dadurch
nicht belastet.
Man kann in mehrere Framebuffer gleichzeitig rendern dank \emph{gl\_Layer}.

\subsection*{Anwendungen:}

\begin{itemize}
  \item Haare, Gras, Fell
  \item es wäre denkbar, ein mesh zu erzeugen bzw. zu erweitern,
    welches als Grundlage für eine Landschaft dient
  \item die Darstellung von selbstständigem Pflanzenwachstum ohne
    CPU-Einsatz (durch Bildung von Dreiecksketten)
  \item die Berechnung von Physikeffekten wie wallender Nebel oder Wasser
  \item Tesselierung vorliegender Geometrien
\end{itemize}


\section*{3. OpenGL state machine}

``OpenGL verwendet eine state machine''

-> je nach Definition von ``state machine'' (die Phasen der
Verarbeitung sind die states vs. die Verarbeitung der Graphikdaten in
den einzelnen Phasen ist nur auf vordefininierte Weise möglich) trifft
dies nur vollständig auf OpenGL <= 1.5 zu, denn ab OpenGL
2.0 gab es Shader

\subsection*{Vorteile}

\begin{itemize}
  \item Anweisungen an den Grafikprozessor können zwischengespeichert werden
  (queued). Dadurch lassen sich verschiedene States definieren und bei Bedarf an
  die GPU abgesetzt werden.
  \item optimale Anpassung der zugrunde liegenden Hardware
  \item einfacher zu Benutzen für den Anwender, aufgrund der
    geringeren Komplexität und der eingeschränkten Möglichkeiten
  \item geringe Hardwareanforderungen
\end{itemize}

\subsection*{Nachteile}

\begin{itemize}
  \item Die gesetzten Werte sind global über den gesamten Code verfügbar und
  können dadurch ungewollt verändert werden.
  \item Es können Objekte nicht direkt angesprochen werden, sondern nur indirekt
  über einen Szenegraphen.
  \item kaum Flexibilität, nur im Rahmen der vorgegebenen
    Einstellungsmöglichkeiten
\end{itemize}

\subsection*{Heute noch gerechtfertigt?}

Der Ansatz der State Machine ist auf heutiger Grafikhardware immer noch
gerechtfertigt, denn durch sie lassen sich mehrere Zustände einer grafischen
Anwendung beschreiben.  Im Falle von OpenGL lassen sich mehrere Zustände
gleichzeitig abbilden in dem sie nacheinander gespeichert werden und daraufhin
an die Grafikhardware geschickt werden.

Silvio: Teilweise ja, meiner Meinung nach ist das ja auch, was der
gegenwärtige Stand der Dinge zeigt. Eine reine StateMachine ist m.E.
heutzutage nicht mehr angemessen, da mit dieser keine adäquaten
Grafikeffekte erzielt werden können, gemessen an den Erwartungen
heutiger Anwender. Das man sich z.B. bei WebGL von Anfang an auf eine
shaderbasierte Implementation festgelegt hat unterstreicht diese
Einschätzung.
Dennoch sind wir noch nicht soweit voran geschritten, das es
geeignetere Konzepte gibt, als die Vorstellung der phasenweise zu
durchlaufenden Graphikpipeline.

\section*{4.}

\begin{center}
\(
T S R T^{-1}
\)
\end{center}

\(
T = \left(
	\begin{array}{ccc}
		1 & 0 & -p_{x} \\
		0 & 1 & -p_{y} \\
		0 & 0 & 1
	\end{array}
\right)
S = \left(
	\begin{array}{ccc}
		1 & 0 & 0 \\
		0 & s_{y} & 0 \\
		0 & 0 & 1
	\end{array}
\right)
R = \left(
	\begin{array}{ccc}
		\cos\alpha & -\sin\alpha & 0 \\
		\sin\alpha & \cos\alpha & 0 \\
		0 & 0 & 1
	\end{array}
\right)
T^{-1} = \left(
	\begin{array}{ccc}
		1 & 0 & p_{x} \\
		0 & 1 & p_{y} \\
		0 & 0 & 1
	\end{array}
\right)
\)

\subsubsection*{Kreis mit Mittlelpunkt im Ursprung des Koordinatensystems:}

\begin{itemize}
  \item der Kreis wird zu einer Ellipse
  \item der Ursprung des Kreises liegt nicht mehr im Mittelpunkt des
Koordinatensystems
  \item der Kreis liegt nun im linken oberen Quadranten des Koordinatensystems
\end{itemize}

\subsubsection*{Kreis mit Mittelpunkt in p}

\begin{itemize}
  \item der Kreis wird zu einer Ellipse
  \item der Mittelpunkt des Kreises liegt noch immer in p
\end{itemize}

\end{document}
