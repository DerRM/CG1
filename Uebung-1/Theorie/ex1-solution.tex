\documentclass[12pt]{scrreprt}

\usepackage[ngerman]{babel}
\usepackage{textcomp}
\usepackage[T1]{fontenc}
\usepackage[utf8]{inputenc}

\usepackage{color,fancyvrb}
\usepackage{lmodern}

\usepackage{graphicx}
\usepackage{amssymb,amsmath}
%\graphicspath{{../latex/}}

%% Quellcode auszeichnung %%
\usepackage{minted}
\makeatletter
\minted@define@extra{label}
\makeatother
\definecolor{bg}{rgb}{0.95,0.95,0.95}
\usemintedstyle{manni}
\newminted{c}{fontfamily=helvetica, fontsize=\scriptsize, tabsize=8,
samepage, frame=single, bgcolor=bg,label=Quellcode,
numbersep=5pt, xrightmargin=0mm, xleftmargin=4mm, linenos}

\usepackage{url}
\usepackage{wrapfig}
\usepackage[font=small,labelfont=bf,textfont=sf]{caption}
\usepackage{placeins}
\usepackage{multirow}
\setlength{\parskip}{3ex plus 2ex minus 2ex}
\widowpenalty=300
\clubpenalty=300

\newcommand{\comment}[1]{}
\setcounter{secnumdepth}{6}

% header and footer
\usepackage{fancyhdr}
\setlength{\headheight}{18pt}
\renewcommand{\headrulewidth}{0.3pt}
\pagestyle{fancyplain}
\renewcommand{\chaptermark}[1]{\markboth{#1}{}}
\fancyhf{}
\cfoot{\thepage}

% make internal and external links
\usepackage{hyperref} % load this at the end of all packages, but
%before the options, not true here at the moment tough

\hypersetup{
%    bookmarks=true,         % show bookmarks bar? - is already true
    unicode=false,          % non-Latin characters in Acrobat’s bookmarks
    pdftoolbar=true,        % show Acrobat’s toolbar?
    pdfmenubar=true,        % show Acrobat’s menu?
    pdffitwindow=false,     % window fit to page when opened
    pdfstartview={FitH},    % fits the width of the page to the window
    pdftitle={CG 1 - Exercise},    % title
    pdfauthor={CG 1 - Gruppe 6},     % author
    pdfsubject={},   % subject of the document
    pdfcreator={CG 1 - Gruppe 6},   % creator of the document
    pdfproducer={Producer}, % producer of the document
    pdfkeywords={CG 1} {Computer Graphics I} {TU Berlin}, % list of keywords
    pdfnewwindow=true,      % links in new window
    colorlinks=false,       % false: boxed links; true: colored links
    linkcolor=red,          % color of internal links
    citecolor=green,        % color of links to bibliography
    filecolor=magenta,      % color of file links
    urlcolor=cyan           % color of external links
}

\begin{document}

\begin{titlepage}
  \begin{center}
    \vspace*{1cm}
    \textbf
    {\emph
      {\Large Computer Graphics I}}\\[1cm]
    \textbf
    {\Large Technische Universität Berlin}\\[.5cm]
    \textbf
    {\Large WS 2012/2013}\\[2cm]
    %\includegraphics[bb=0 0 60 40]{./TUB_Logo.png}\\[0.9cm]
    \vfill
    {\Large Gruppe 11}\\[1cm]
    Christopher Sierigk\\
    Bernd Loeber\\
    Silvio Hädrich\\
    \vspace*{2cm}
    \today
  \end{center}
  \vspace*{2cm}
\end{titlepage}

\pagenumbering{arabic}

\chapter*{Aufgabe 1 - Praktischer Teil}

siehe Quellcode

\chapter*{Aufgabe 2 - Theoriefragen}
\section*{1.}

\subsection*{a.}
Beschreiben Sie die folgende Transformation des lokalen Koordinatensystems an p = (px,py) (blau) mit
elementaren 3x3 Transformationsmatrizen (1 Punkt). Welchen Effekt hat die Transformation auf einen
Kreis, der am Ursprung zentriert ist? Welchen Effekt hat die Transformation auf einen Kreis, der an p
zentriert ist?


\begin{center}
\(
T S R T^{-1}
\)
\end{center}

\(
T = \left(
	\begin{array}{ccc}
		1 & 0 & -p_{x} \\
		0 & 1 & -p_{y} \\
		0 & 0 & 1
	\end{array}
\right)
S = \left(
	\begin{array}{ccc}
		1 & 0 & 0 \\
		0 & s_{y} & 0 \\
		0 & 0 & 1
	\end{array}
\right)
R = \left(
	\begin{array}{ccc}
		\cos\alpha & -\sin\alpha & 0 \\
		\sin\alpha & \cos\alpha & 0 \\
		0 & 0 & 1
	\end{array}
\right)
T^{-1} = \left(
	\begin{array}{ccc}
		1 & 0 & p_{x} \\
		0 & 1 & p_{y} \\
		0 & 0 & 1
	\end{array}
\right)
\)

\subsubsection*{Kreis mit Mittlelpunkt im Ursprung des Koordinatensystems:}

\begin{itemize}
  \item der Kreis wird zu einer Ellipse
  \item der Ursprung des Kreises liegt nicht mehr im Mittelpunkt des
Koordinatensystems
  \item der Kreis liegt nun im linken oberen Quadranten des Koordinatensystems
\end{itemize}

\subsubsection*{Kreis mit Mittelpunkt in p}

\begin{itemize}
  \item der Kreis wird zu einer Ellipse
  \item der Mittelpunkt des Kreises liegt noch immer in p
\end{itemize}


\subsection*{b.}

Leiten Sie schrittweise eine Formel her, die einen Vektor v um eine beliebige Achse, gegeben durch den
Einheitsvektor k, mit dem Winkel \(\theta\) rotiert. Dabei sollen ausschliesslich Vektoroperationen, jedoch keine
Matrixoperationen, verwendet werden. Folgende Schritte sind dazu notwendig:
1. Bestimmen Sie die Projektion vk des Vektor v auf Vektor k. (0,5 Punkte)

\[
 v_k = {v * k \over k^2} * k
\]

2. Bestimmen Sie zwei orthogonale Basisvektoren v1 und v2, die die Fläche senkrecht zu Vektor
aufspannen. (0,5 Punkte)

\(v_1\) bestimmen wir über das Skalarprodukt. Es gilt:
\[
v_k * v_1 = |v_k| * |v_1| * \cos\frac{\pi}{2} = 0
\]
\[
v_{k_1}v_{1_1} + v_{k_2}v_{1_2} + v_{x_3}v_{1_3} = 0
\]

\(v_2\) bestimmen wir über das Kreuzprodukt. Es gilt:
\[
v_2 = v_1 \times v_k
\]

3. Bestimmen Sie die um den Winkel \(\theta\) rotierten Vektoren \(v_1', v_2' und v_k'\). (0,5 Punkte)

Dazu kann man die Schreibweise als Polarkoordinaten nutzen:


4. Fassen Sie alle Teilschritte zu einer Formel in Abhängigkeit von v, k und \(\theta\) zusammen, die den
rotierten Vektor w bestimmt. (0,5 Punkte)



\subsection*{c.}

Gegeben ist die homogene Transformationsmatrix M als Produkt einer Translationsmatrix T und einer
Rotationsmatrix R, M = TR. Bestimmen Sie die Inverse der Transformationsmatrix (M-1) unter
Berücksichtigung bzw. Ausnutzung der Eigenschaften der gegebenen Matrizen.

geg.: M = TR; I = Identity
ges.: \[M^{-1} \\
 \\
TT^{-1} = I \\
RR^{-1} = I \\
M^{-1} = R^{-1} * T^{-1}
\]

Die Inverse einer Translation erhaelt man durch Invertieren des Translationsvektors (Vorzeichen umdrehen).
Die Inverse einer Rotation erhaelt man durch Transponieren.

\[
 M^{-1} = \begin{bmatrix}
       \cos \alpha & \sin\alpha & 0 & -T_1 \\[0.3em]
       -\sin\alpha & \cos\alpha & 0 & -T_2 \\[0.3em]
       0           & 0          & 0 & -T_3 \\[0.3em]
       0           & 0          & 0 & 1
     \end{bmatrix}
\]

\subsection*{d}

Für welche Kombinationen zweier Transformationen (Rotation, isotrope/anisotrope Skalierung,
Translation) ist die Reihenfolge der Anwendung irrelevant? Es gibt 10 mögliche Kombinationen. Erstellen
Sie eine Tabelle.


R = Rotation, T = Translation,  iS/aS = isotrope/anisotrope Skalierung

Ich hab mir ueberlegt, das man alles mit einer Rotation nicht vertauschen kann, bin mir aber nicht ganz sicher

RiS entspricht iSR = falsch\\
RT entspricht TR   = falsch\\
iST entspricht TiS = wahr\\
TaS entspricht aST = wahr\\
RaS entspricht aSR = falsch\\


\subsection*{e}

In der Programmieraufgabe wird double buffering verwendet (siehe Funktion display() in
context.c). Was passiert beim double buffering und wieso wird es in dieser Aufgabe verwendet?

-> man hat 2 Framebuffer, waehrend einer auf dem Display angezeigt wird, wird in dem 2. Buffer der naechste Fram gerendert. Anschliessend werden beide Buffer vertauscht. Ermoeglicht fluessiges anzeigen von Bewegungen (des Roboters).



\begin{itemize}
  \item Haare, Gras, Fell
  \item es wäre denkbar, ein mesh zu erzeugen bzw. zu erweitern,
\end{itemize}


``OpenGL verwendet eine state machine''



\end{document}
