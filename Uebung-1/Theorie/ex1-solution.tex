\documentclass[12pt]{scrreprt}

\usepackage[ngerman]{babel}
\usepackage{textcomp}
\usepackage[T1]{fontenc}
\usepackage[utf8]{inputenc}

\usepackage{color,fancyvrb}
\usepackage{lmodern}

\usepackage{graphicx}
\usepackage{amssymb,amsmath}
%\graphicspath{{../latex/}}

%% Quellcode auszeichnung %%
\usepackage{minted}
\makeatletter
\minted@define@extra{label}
\makeatother
\definecolor{bg}{rgb}{0.95,0.95,0.95}
\usemintedstyle{manni}
\newminted{c}{fontfamily=helvetica, fontsize=\scriptsize, tabsize=8,
samepage, frame=single, bgcolor=bg,label=Quellcode,
numbersep=5pt, xrightmargin=0mm, xleftmargin=4mm, linenos}

\usepackage{url}
\usepackage{wrapfig}
\usepackage[font=small,labelfont=bf,textfont=sf]{caption}
\usepackage{placeins}
\usepackage{multirow}
\setlength{\parskip}{3ex plus 2ex minus 2ex}
\widowpenalty=300
\clubpenalty=300

\newcommand{\comment}[1]{}
\setcounter{secnumdepth}{6}

% header and footer
\usepackage{fancyhdr}
\setlength{\headheight}{18pt}
\renewcommand{\headrulewidth}{0.3pt}
\pagestyle{fancyplain}
\renewcommand{\chaptermark}[1]{\markboth{#1}{}}
\fancyhf{}
\cfoot{\thepage}

% make internal and external links
\usepackage{hyperref} % load this at the end of all packages, but
%before the options, not true here at the moment tough

\hypersetup{
%    bookmarks=true,         % show bookmarks bar? - is already true
    unicode=false,          % non-Latin characters in Acrobat’s bookmarks
    pdftoolbar=true,        % show Acrobat’s toolbar?
    pdfmenubar=true,        % show Acrobat’s menu?
    pdffitwindow=false,     % window fit to page when opened
    pdfstartview={FitH},    % fits the width of the page to the window
    pdftitle={CG 1 - Exercise},    % title
    pdfauthor={CG 1 - Gruppe 6},     % author
    pdfsubject={},   % subject of the document
    pdfcreator={CG 1 - Gruppe 6},   % creator of the document
    pdfproducer={Producer}, % producer of the document
    pdfkeywords={CG 1} {Computer Graphics I} {TU Berlin}, % list of keywords
    pdfnewwindow=true,      % links in new window
    colorlinks=false,       % false: boxed links; true: colored links
    linkcolor=red,          % color of internal links
    citecolor=green,        % color of links to bibliography
    filecolor=magenta,      % color of file links
    urlcolor=cyan           % color of external links
}

\begin{document}

\begin{titlepage}
  \begin{center}
    \vspace*{1cm}
    \textbf
    {\emph
      {\Large Computer Graphics I}}\\[1cm]
    \textbf
    {\Large Technische Universität Berlin}\\[.5cm]
    \textbf
    {\Large WS 2012/2013}\\[2cm]
    %\includegraphics[bb=0 0 60 40]{./TUB_Logo.png}\\[0.9cm]
    \vfill
    {\Large Gruppe 11}\\[1cm]
    Christopher Sierigk\\
    Bernd Loeber\\
    Silvio Hädrich\\
    \vspace*{2cm}
    \today
  \end{center}
  \vspace*{2cm}
\end{titlepage}

\pagenumbering{arabic}

\chapter*{Excercise 1 - 2. Theoriefragen}

\subsection*{a.}
Beschreiben Sie die folgende Transformation des lokalen Koordinatensystems an p = (px,py) (blau) mit
elementaren 3x3 Transformationsmatrizen (1 Punkt). Welchen Effekt hat die Transformation auf einen
Kreis, der am Ursprung zentriert ist? Welchen Effekt hat die Transformation auf einen Kreis, der an p
zentriert ist?

Es findet eine anisotrope Skalierung mit $s_y = 2$ sowie eine Rotation mit $\alpha = 45^\circ$ statt.
Zuvor muss um $-p_x/-p_y$ verschoben werden und diese Verschiebung zuguterletzt wieder in umgekehrter Richtung erfolgen.

\begin{center}
\(
T S R T^{-1}
\)
\end{center}

\(
T = \left(
	\begin{array}{ccc}
		1 & 0 & -p_{x} \\
		0 & 1 & -p_{y} \\
		0 & 0 & 1
	\end{array}
\right)
S = \left(
	\begin{array}{ccc}
		1 & 0 & 0 \\
		0 & s_{y} & 0 \\
		0 & 0 & 1
	\end{array}
\right)
R = \left(
	\begin{array}{ccc}
		\cos\alpha & -\sin\alpha & 0 \\
		\sin\alpha & \cos\alpha & 0 \\
		0 & 0 & 1
	\end{array}
\right)
T^{-1} = \left(
	\begin{array}{ccc}
		1 & 0 & p_{x} \\
		0 & 1 & p_{y} \\
		0 & 0 & 1
	\end{array}
\right)
\)

\subsubsection*{Kreis mit Mittelpunkt im Ursprung des Koordinatensystems:}

\begin{itemize}
  \item der Kreis wird zu einer Ellipse
  \item der Ursprung des Kreises liegt nicht mehr im Mittelpunkt des
Koordinatensystems
  \item der Kreis liegt nun im linken oberen Quadranten des Koordinatensystems
\end{itemize}

\begin{center}
        \includegraphics[width=0.5\textwidth]{rotation1}
    \captionof{figure}{Mittelpunkt im Ursprung}
\end{center}

\subsubsection*{Kreis mit Mittelpunkt in p}

\begin{itemize}
  \item der Kreis wird zu einer Ellipse
  \item der Mittelpunkt des Kreises liegt noch immer in p
\end{itemize}

\begin{center}
        \includegraphics[width=0.5\textwidth]{rotation2}
    \captionof{figure}{Mittelpunkt in p}
\end{center}

\pagebreak[4]

\subsection*{b.}

Leiten Sie schrittweise eine Formel her, die einen Vektor $v$ um eine beliebige Achse, gegeben durch den
Einheitsvektor $k$, mit dem Winkel $\theta$ rotiert. Dabei sollen ausschließlich Vektoroperationen, jedoch keine
Matrixoperationen, verwendet werden. Folgende Schritte sind dazu notwendig:

1. Bestimmen Sie die Projektion $v_k$ des Vektor $v$ auf Vektor $k$. (0,5 Punkte)

\[
	v_k = {v \cdot k \over \|k\|^2}~k
\]

lässt sich auch schreiben als:

\[
	v_k = (v \cdot k)k, ~da ~\| k \| = 1
\]

2. Bestimmen Sie zwei orthogonale Basisvektoren $v_1$ und $v_2$, die die Fläche senkrecht zu Vektor $v_k$
aufspannen. (0,5 Punkte)

$v_1$ bestimmen wir, indem wir uns die Projektion von $v$ auf $k$ zu Nutze machen und somit der Vektor $v_1$ von $v_k$ nach $v$ demnach senkrecht zu $v_k$ sein muss. Es gilt:

\[
v_1 = v - v_k
\]

$v_2$ bestimmen wir über das Kreuzprodukt. Es gilt:

\begin{align*}
	v_2 &= v_1 \times k bzw.\\
	v_2 &= v \times k
\end{align*}

weil $v$ und $k$ eine Ebene aufspannen die senkrecht zu $v_2$ sein muss.

3. Bestimmen Sie die um den Winkel $\theta$ rotierten Vektoren $v_1'$, $v_2'$ und $v_k'$. (0,5 Punkte)

Da um den Vektor $k$ rotiert werden soll und $v_k$ auf $liegt$ ergibt sich sofort

\[
	v_k' = v_k
\]

Da nun $v_1$ und $v_2$ in einer Ebene liegen und um den Schnittpunkt der beiden gedreht wird, kann man die Rotation der beiden Vektoren als Rotation im Polarkreis darstellen. Dabei ergibt sich:

\begin{align*}
	v_1' &= \cos{\theta} v_1 + \sin{\theta} v_2\\
	v_2' &= -\sin{\theta} v_2 + \cos{\theta} v_2
\end{align*}

4. Fassen Sie alle Teilschritte zu einer Formel in Abhängigkeit von $v$, $k$ und $\theta$ zusammen, die den
rotierten Vektor $w$ bestimmt. (0,5 Punkte)

$w$ lässt sich nun in Abhängigkeit von $v$, $k$ und $\theta$ folgendermaßen formulieren:

\begin{align*}
	w &= v_k + v_1'\\
	w &= (v \cdot k)k + \cos{\theta} v_1 + \sin{\theta} v_2\\
	w &= (v \cdot k)k + \cos{\theta} (v - (v \cdot k)k) + \sin{\theta} (v \times k)\\
	w &= (v \cdot k)k + \cos{\theta}v - \cos{theta}(v \cdot k)k + \sin{\theta}(v \times k)\\
	w &= (1 - \cos{\theta})(v \cdot k)k + \cos{\theta}v + \sin{\theta}(k \times v)
\end{align*}

\subsection*{c.}

Gegeben ist die homogene Transformationsmatrix M als Produkt einer Translationsmatrix T und einer
Rotationsmatrix R, M = TR. Bestimmen Sie die Inverse der Transformationsmatrix (M-1) unter
Berücksichtigung bzw. Ausnutzung der Eigenschaften der gegebenen Matrizen.\\
\\
geg.: M = TR; I = Identity\\
ges.: $M^{-1}$ \\

\begin{align*}
	TT^{-1} &= I \\
	RR^{-1} &= I \\
	M^{-1} &= R^{-1} ~ T^{-1}
\end{align*}

Die Inverse einer Translation erhält man durch das Invertieren des Translationsvektors (Vorzeichen umdrehen).
Die Inverse einer Rotation erhält man durch das Transponieren der 3x3 Rotationsmatrix.

\[
 M^{-1} = \begin{bmatrix}
       \cos \alpha & \sin\alpha & 0 & -T_1 \\[0.3em]
       -\sin\alpha & \cos\alpha & 0 & -T_2 \\[0.3em]
       0           & 0          & 0 & -T_3 \\[0.3em]
       0           & 0          & 0 & 1
     \end{bmatrix}
\]

\subsection*{d}

Für welche Kombinationen zweier Transformationen (Rotation, isotrope/anisotrope Skalierung,
Translation) ist die Reihenfolge der Anwendung irrelevant? Es gibt 10 mögliche Kombinationen. Erstellen
Sie eine Tabelle.

R = Rotation, T = Translation,  iS/aS = isotrope/anisotrope Skalierung\\
\\
RiS entspricht iSR = wahr\\
RT entspricht TR   = falsch\\
iST entspricht TiS = wahr\\
TaS entspricht aST = wahr\\
RaS entspricht aSR = falsch\\

\subsection*{e}

In der Programmieraufgabe wird \emph{double buffering} verwendet (siehe Funktion display() in
context.c). Was passiert beim \emph{double buffering} und wieso wird es in dieser Aufgabe verwendet?

Während des Renderings stehen 2 Framebuffer zur Verfügung. Das heißt es kann in 2 Buffer geschrieben werden um Bildinformationen zu speichern. Während der Inhalt des ersten Buffers auf dem Display angezeigt wird, wird in dem 2. Buffer der nächste Frame gerendert. Anschließend werden beide Buffer vertauscht. Dies ermöglicht ein flüssiges Anzeigen von Bewegungen und Animationen, da die benötigten Informationen zeitnah zur Verfügung stehen. Diese Technik kann aber auch Nachteile mit sich bringen. Sobald der Tauschvorgang nicht mehr mit der Frequenz des Monitors (Vsync) übereinstimmt, können Artefakte, das sogenannte Tearing, entstehen. Am Beispiel der Übung lässt sich der Roboter ruckelfrei hin- und herbewegen.

\end{document}
