\documentclass[12pt]{scrreprt}

\usepackage[ngerman]{babel}
\usepackage{textcomp}
\usepackage[T1]{fontenc}
\usepackage[utf8]{inputenc}

\usepackage{color,fancyvrb}
\usepackage{lmodern}

\usepackage{graphicx}
\usepackage{amssymb,amsmath}
%\graphicspath{{../latex/}}

%% Quellcode auszeichnung %%
\usepackage{minted}
\makeatletter
\minted@define@extra{label}
\makeatother
\definecolor{bg}{rgb}{0.95,0.95,0.95}
\usemintedstyle{manni}
\newminted{c}{fontfamily=helvetica, fontsize=\scriptsize, tabsize=8,
samepage, frame=single, bgcolor=bg,label=Quellcode,
numbersep=5pt, xrightmargin=0mm, xleftmargin=4mm, linenos}

\usepackage{url}
\usepackage{wrapfig}
\usepackage[font=small,labelfont=bf,textfont=sf]{caption}
\usepackage{placeins}
\usepackage{multirow}
\setlength{\parskip}{3ex plus 2ex minus 2ex}
\widowpenalty=300
\clubpenalty=300

\newcommand{\comment}[1]{}
\setcounter{secnumdepth}{6}

% header and footer
\usepackage{fancyhdr}
\setlength{\headheight}{18pt}
\renewcommand{\headrulewidth}{0.3pt}
\pagestyle{fancyplain}
\renewcommand{\chaptermark}[1]{\markboth{#1}{}}
\fancyhf{}
\cfoot{\thepage}

% make internal and external links
\usepackage{hyperref} % load this at the end of all packages, but
%before the options, not true here at the moment tough

\hypersetup{
%    bookmarks=true,         % show bookmarks bar? - is already true
    unicode=false,          % non-Latin characters in Acrobat’s bookmarks
    pdftoolbar=true,        % show Acrobat’s toolbar?
    pdfmenubar=true,        % show Acrobat’s menu?
    pdffitwindow=false,     % window fit to page when opened
    pdfstartview={FitH},    % fits the width of the page to the window
    pdftitle={CG 1 - Exercise},    % title
    pdfauthor={CG 1 - Gruppe 6},     % author
    pdfsubject={},   % subject of the document
    pdfcreator={CG 1 - Gruppe 6},   % creator of the document
    pdfproducer={Producer}, % producer of the document
    pdfkeywords={CG 1} {Computer Graphics I} {TU Berlin}, % list of keywords
    pdfnewwindow=true,      % links in new window
    colorlinks=false,       % false: boxed links; true: colored links
    linkcolor=red,          % color of internal links
    citecolor=green,        % color of links to bibliography
    filecolor=magenta,      % color of file links
    urlcolor=cyan           % color of external links
}

\begin{document}

\begin{titlepage}
  \begin{center}
    \vspace*{1cm}
    \textbf
    {\emph
      {\Large Computer Graphics I}}\\[1cm]
    \textbf
    {\Large Technische Universität Berlin}\\[.5cm]
    \textbf
    {\Large WS 2012/2013}\\[2cm]
    %\includegraphics[bb=0 0 60 40]{./TUB_Logo.png}\\[0.9cm]
    \vfill
    {\Large Gruppe 11}\\[1cm]
    Christopher Sierigk\\
    Bernd Loeber\\
    Silvio Hädrich\\
    \vspace*{2cm}
    \today
  \end{center}
  \vspace*{2cm}
\end{titlepage}

\pagenumbering{arabic}

\chapter*{Aufgabe 1 - Praktischer Teil}

siehe Quellcode

\chapter*{Aufgabe 2 - Theoriefragen}
\section*{1.}
Das Problem entsteht dadurch, dass bei der linearen Interpolation nicht die perspektivische Verzerrung berücksichtigt wird. Das bedeutet, dass eine Strecke die weiter entfernt ist länger ist in der Projektion als ein Strecke die näher am Betrachter liegt. 
Diese lineare Interpolation findet nach der Projektion in den Raum statt, wird also demnach nicht mehr bei der Berechnung berücksichtigt.
\section*{2.}
Mipmaps werden dazu benutzt um Artefakte in Aliasing zu minimieren. Es werden mehrere Auflösungen des gleichen Bildes in der Mipmap gespeichert. Die erwähnten Artefakte kommen z.B. dazu zustande, wenn an ein texturiertes Objekt herangezoomt wird. Je näher man dem Objekt kommt, desto eher kann es passieren, dass die Textur verpixelt erscheint. Wird aus dem Objekt herausgezoomt, kann es zu Aliasing kommen, wodurch es erscheint, dass die Textur zu flimmern beginnt. Diese Effekte werden durch Mipmaps verringert. Außerdem wird dadurch der Texturlookup verkürzt, da schon alle Zoomstufen der Textur in der Mipmap enthalten sind.
\section*{3.}
Die Familie der Objekte, die sich bijektiv auf eine Kugel abbilden lassen, sind zum einen alle konvexen Objekte, die den Mittelpunkt der Kugel innerhalb haben (also alle). Dies gilt zudem auch für diejenigen konkaven Objekte, die ebenfalls Kugelmittelpunkt innerhalb haben, was bedeutet, dass dies nur für eine Teilmenge der konkaven Objekte gilt.
% Quelle: http://home.arcor.de/whqsk/gdv5/CG5%20Textaufgaben.txt
\section*{4.}
Die Verzerrungen kommen dadurch zu Stande, da die Environment Map auf eine Kugel abgebildet wird. Da eine Kugel eine runde Oberfläche besitzt, kommt es zu Verzerrungen, wenn diese auf eine planare Ebene abgebildet werden muss. Vorallem sind diese Verzerrungen am Rand der Textur zu erkennen.
\section*{5.}
\subsection*{a)}
\subsection*{b)}
\end{document}
