\documentclass[12pt]{scrreprt}

\usepackage[ngerman]{babel}
\usepackage{textcomp}
\usepackage[T1]{fontenc}
\usepackage[utf8]{inputenc}

\usepackage{color,fancyvrb}
\usepackage{lmodern}

\usepackage{graphicx}
\usepackage{amssymb,amsmath}
%\graphicspath{{../latex/}}

%% Quellcode auszeichnung %%
\usepackage{minted}
\makeatletter
\minted@define@extra{label}
\makeatother
\definecolor{bg}{rgb}{0.95,0.95,0.95}
\usemintedstyle{manni}
\newminted{c}{fontfamily=helvetica, fontsize=\scriptsize, tabsize=8,
samepage, frame=single, bgcolor=bg,label=Quellcode,
numbersep=5pt, xrightmargin=0mm, xleftmargin=4mm, linenos}

\usepackage{url}
\usepackage{wrapfig}
\usepackage[font=small,labelfont=bf,textfont=sf]{caption}
\usepackage{placeins}
\usepackage{multirow}
\setlength{\parskip}{3ex plus 2ex minus 2ex}
\widowpenalty=300
\clubpenalty=300

\newcommand{\comment}[1]{}
\setcounter{secnumdepth}{6}

% header and footer
\usepackage{fancyhdr}
\setlength{\headheight}{18pt}
\renewcommand{\headrulewidth}{0.3pt}
\pagestyle{fancyplain}
\renewcommand{\chaptermark}[1]{\markboth{#1}{}}
\fancyhf{}
\cfoot{\thepage}

% make internal and external links
\usepackage{hyperref} % load this at the end of all packages, but
%before the options, not true here at the moment tough

\hypersetup{
%    bookmarks=true,         % show bookmarks bar? - is already true
    unicode=false,          % non-Latin characters in Acrobat’s bookmarks
    pdftoolbar=true,        % show Acrobat’s toolbar?
    pdfmenubar=true,        % show Acrobat’s menu?
    pdffitwindow=false,     % window fit to page when opened
    pdfstartview={FitH},    % fits the width of the page to the window
    pdftitle={CG 1 - Exercise},    % title
    pdfauthor={CG 1 - Gruppe 6},     % author
    pdfsubject={},   % subject of the document
    pdfcreator={CG 1 - Gruppe 6},   % creator of the document
    pdfproducer={Producer}, % producer of the document
    pdfkeywords={CG 1} {Computer Graphics I} {TU Berlin}, % list of keywords
    pdfnewwindow=true,      % links in new window
    colorlinks=false,       % false: boxed links; true: colored links
    linkcolor=red,          % color of internal links
    citecolor=green,        % color of links to bibliography
    filecolor=magenta,      % color of file links
    urlcolor=cyan           % color of external links
}

\begin{document}

\begin{titlepage}
  \begin{center}
    \vspace*{1cm}
    \textbf
    {\emph
      {\Large Computer Graphics I}}\\[1cm]
    \textbf
    {\Large Technische Universität Berlin}\\[.5cm]
    \textbf
    {\Large WS 2012/2013}\\[2cm]
    %\includegraphics[bb=0 0 60 40]{./TUB_Logo.png}\\[0.9cm]
    \vfill
    {\Large Gruppe 11}\\[1cm]
    Christopher Sierigk\\
    Bernd Loeber\\
    Silvio Hädrich\\
    \vspace*{2cm}
    \today
  \end{center}
  \vspace*{2cm}
\end{titlepage}

\pagenumbering{arabic}

\chapter*{Aufgabe 1 - Praktischer Teil}

siehe Quellcode

\chapter*{Aufgabe 2 - Theoriefragen}
\section*{1.}
Der Grund für das Clipping am kanonischen Sichtvolumen in OpenGL ist Performance. Durch das Clipping werden alle Teile der Geometrie abgeschnitten, die außerhalb des sichtbaren Bereichs liegen. Dies hat den Grund beim Rendern Rechenarbeit zu sparen, da es wenig Sinn macht Objekte bzw. Teile von Objekten zu rendern, die eindeutig nicht gesehen werden können.
\section*{2.}
Da ein 3dimensionales kanonisches Sichtvolumen 6 Seiten hat, besteht der Outcode aus 6 Ziffern die entweder 1 oder 0 sein können. Die Seiten außerhalb des Volumens kriegen jeweils einen eindeutigen Code zugeordnet:

\begin{align*}
	000001& &top\\
	000010& &bottom\\
	000100& &left\\
	001000& &right\\
	010000& &front\\
	100000& &back
\end{align*}

Wenn sich 2 Seiten an einer Kante schneiden werden die jeweiligen Codes verodert:

\begin{align*}
	000101& &top-left\\
	001001& &top-right\\
	010001& &top-front\\
	100001& &top-back\\
	000110& &bottom-left\\
	001010& &bottom-right\\
	010010& &bottom-front\\
	100010& &bottom-back\\
	010100& &front-left\\
	011000& &front-right\\
	100100& &back-left\\
	101000& &back-right
\end{align*}

Treffen sich die Kanten an einem Eckpunkt bzw. schneiden sich 3 Seiten an einem gemeinsamen Punkt entstehen wiederum verschiedene Outcodes:

\begin{align*}
	0100101& &top-left-front\\
	0110001& &top-right-front\\
	1001001& &top-left-back\\
	1010001& &top-right-back\\
	0100110& &bottom-left-front\\
	0110010& &bottom-right-front\\
	1001010& &bottom-left-back\\
	1010010& &bottom-right-back
\end{align*}

Der Raum innerhalb des kanonische Volumens bekommt den eindeutigen Code $000000$zugewiesen. Als Primitive zum unterteilen der einzelnen Halbräume eignen sich im 3D-Fall die Ebenen.

\section*{3.}
\section*{4.}
Wenn wir zum Clipping den Sutherland Hodgman Algorithmus verwenden, lässt sich ein beliebiges Polygon anhand eines konvexen Polygons clippen. Die Komplexität, die dabei entsteht ist dabei $O(n)$. Wobei sich das $n$ auf die Anzahl der Kanten des konvexen Polygons bezieht an welchem das beliebige Polygon geclippt werden soll. Dies rührt daher, dass der Algorithmus zum Clippen alle vorhandenen Kanten durchgeht und die Schnittpunkte mit dem beliebigen Polygon berechnet.
\section*{5.}
Da üblicherweise alle unnötigen Geometrien außerhalb des kanonische Sichtvolumens geclippt werden, würde es wenig Sinn machen für nicht sichtbare Dreiecke zu berechnen in welche Richtung diese orientiert sind und deren Sichtbarkeit zu bestimmen.
\section*{6.}
\subsection*{(a)}
Wir nehmen einen beliebigen Vektor $\vec{v}$ aus der Ebene und die Normale $\vec{n}$ der Ebene. Dabei lässt sich die Ebene wie folgt beschreiben:

\[
	\vec{n} \cdot \vec{v} = n_x \cdot v_x + n_y \cdot v_y + n_z \cdot v_z = 0
\] 
\subsection*{(b)}
Um die Transformationsmatrix für die Normale $n$ herzuleiten nutzen wir die Gleichung aus (a). Da diese gilt, muss auch folgendes für eine beliebige affine Transformation $T$ gelten:

\begin{align*}
	\vec{v'} &= T \vec{v}\\
	\vec{n'} \cdot \vec{v'} = \vec{n'} \cdot T \vec{v} &= 0 
\end{align*}

Schreiben wir diese Gleichung nun aus, erhalten wir:

\begin{align*}
	n'_x~ (t_{11} v_x + t_{12} v_y + t_{13} v_z + t_{14}) &~+\\
	n'_y~ (t_{21} v_x + t_{22} v_y + t_{23} v_z + t_{24}) &~+\\
	n'_z~ (t_{31} v_x + t_{32} v_y + t_{33} v_z + t_{34}) &~+\\
	0~ (t_{41} v_x + t_{42} v_y + t_{43} v_z + t_{44}) &= 0\\
\end{align*}

Demnach gilt auch folgendes, wenn wir nach den Komponenten von $\vec{v'}$ ausklammern:

\begin{align*}
	v_x~(t_{11} n'_x + t_{21} n'_y + t_{31} n'_z + t_{41} 0 ) &~+\\
	v_y~(t_{12} n'_x + t_{22} n'_y + t_{32} n'_z + t_{42} 0) &~+\\
	v_z~(t_{13} n'_x + t_{23} n'_y + t_{33} n'_z + t_{43} 0) &~+\\
	1~(t_{14} n'_x + t_{24} n'_y + t_{34} n'_z + t_{44} 0) &= 0
\end{align*}

Wenn man dies nun kompakt schreibt:

\[
	\vec{v} \cdot
	\begin{pmatrix}
		t_{11} & t_{21} & t_{31} & t_{41}\\
		t_{12} & t_{22} & t_{32} & t_{42}\\
		t_{13} & t_{23} & t_{33} & t_{43}\\
		t_{14} & t_{24} & t_{34} & t_{44}
	\end{pmatrix}
	\vec{n'} = 0
\]

sieht man sofort, dass die Matrix $T^T$ die transponierte zu $T$ ist, wodurch gilt:

\[
	\vec{n'} = T^T \vec{n}
\]

\subsection*{(c)}
Das Transformationsgesetz bei einer Normalen gilt aus dem Grund nicht, da eine Normale nicht durch eine Punkt zu Punkt Beziehung beschrieben werden kann. Das bedeutet, es lässt sich keine Normale nur aus 2 gegebenen Punkten bestimmen und somit verhält sich eine Normale auch nicht wie ein zu transformierender Punkt.
\subsection*{(d)}
Das hergeleitete Transformationsgesetz für eine beliebige Normale gilt selbst bei gekrümmten Flächen. Um dort die Normale zu erhalten muss zunächst eine Tangentialebene an die gewünschte Stelle angelegt werden. Dadurch lässt sich die Normale bestimmen und wie gehabt transformieren.
\end{document}
